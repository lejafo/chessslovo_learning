\documentclass{article}
\usepackage[utf8]{inputenc}
\usepackage[russian]{babel}

\usepackage{caption}

\usepackage{skak}
\usepackage{xskak}
\usepackage{chessboard}

\captionsetup[figure]{labelfont={bf,bf},textfont={bf,bf}}

% \captionsetup{
%   format=plain,
%   justification=raggedright,
%   singlelinecheck=false,
%   font=small,
%   font={bf,bf} % This covers labelfont and textfont
%   labelsep=pipe,
%   figurename=Figure
% }

\addto\captionsenglish{%
    \renewcommand{\figurename}{}}

\setlength{\abovecaptionskip}{-10pt plus 3pt minus 2pt}
% \setlength{\belowcaptionskip}{-15pt plus 3pt minus 2pt}

% \let\savedchessboard\chessboard
% \renewcommand{\chessboard}{%
%     \begin{figure}[h]%
%         \centering
%         \caption{}
%         \savedchessboard
%     \end{figure}
% }

\newenvironment{nscenter}
 {\parskip=0pt\par\nopagebreak\centering}
 {\par\noindent\ignorespacesafterend}

\newcounter{taskno}
\stepcounter{taskno}
\newcounter{subtaskno}[taskno]

\newcommand{\task}[0]{\par\noindent\textbf{Задача\,\arabic{taskno}.} \stepcounter{taskno} \stepcounter{subtaskno}}
\newcommand{\taskk}[0]{\par\noindent\textbf{Задача\,\arabic{taskno}*.} \stepcounter{taskno} \stepcounter{subtaskno}}
\newcommand{\taskkk}[0]{\par\noindent\textbf{Задача\,\arabic{taskno}**.} \stepcounter{taskno} \stepcounter{subtaskno}}


\begin{document}

\task Найдите выигрышный ход за белых. Ход белых.
\begin{nscenter}
    \chessboard[
        showmover=true,
        % mover=b,
        borderleft=true,
        bordertop=true,
        borderbottom=true,
        labeltoplift=0.4em,
        labeltop=false,
        labelbottom=true,
        labelleft=true,
        labelright=false,
        % printarea=a1-h5,
        %markfields=\fields,
        %markareas=\areas,
        setpieces={Kg1, Qd1, Re1, Ba5, Bh3, Nc3, Nc4, Pb3, Pc2, Pe4, Pf2, Pg3, Ph2},
        addpieces={ke8, qc6, rh8, bb7, bf8, nc5, nf6, pa6, pd6, pe5, pf7, pg6, ph7}
    ]
\end{nscenter}
\vspace{4pt}

\task Каким образом белым поймать черного коня? Ход белых.
\begin{nscenter}
    \chessboard[
        showmover=true,
        % mover=b,
        borderleft=true,
        bordertop=true,
        borderbottom=true,
        labeltoplift=0.4em,
        labeltop=false,
        labelbottom=true,
        labelleft=true,
        labelright=false,
        % printarea=a1-h5,
        %markfields=\fields,
        %markareas=\areas,
        setpieces={Kf4, Bb3, Pa3, Pb2, Pc2, Pd3, Ph3},
        addpieces={ke7, nh2, pa7, pb5, pf5, pg6, ph7}
    ]
\end{nscenter}
\vspace{4pt}

\task Подумайте над выигрышной стратегией за белых при их ходе.
\begin{nscenter}
    \chessboard[
        showmover=true,
        borderleft=true,
        bordertop=true,
        borderbottom=true,
        labeltoplift=0.4em,
        labeltop=false,
        labelbottom=true,
        labelleft=true,
        labelright=false,
        setpieces={Kd5, Rh6, Pe4, Pf3},
        addpieces={kd7, bc5, pd6, pe5, pf4}
    ]
\end{nscenter}
\vspace{4pt}

\task Подумайте над выигрышной стратегией за белых при их ходе.
\begin{nscenter}
    \chessboard[
        showmover=true,
        borderleft=true,
        bordertop=true,
        borderbottom=true,
        labeltoplift=0.4em,
        labeltop=false,
        labelbottom=true,
        labelleft=true,
        labelright=false,
        setpieces={Kf5, Bd5, Pg2},
        addpieces={kh4, bg8, pf7}
    ]
\end{nscenter}
\vspace{4pt}

\task Каким образом белым выиграть? Ход белых.
\begin{nscenter}
    \chessboard[
        showmover=true,
        borderleft=true,
        bordertop=false,
        borderbottom=true,
        labeltoplift=0.4em,
        labeltop=false,
        labelbottom=true,
        labelleft=true,
        labelright=false,
        printarea=a1-h5,
        %markfields=\fields,
        %markareas=\areas,
        setpieces={Kg5, Qg1},
        addpieces={ke2, pd2}
    ]
\end{nscenter}
\vspace{4pt}

\task Могут ли белые выиграть? Ход белых.
\begin{nscenter}
    \chessboard[
        showmover=true,
        borderleft=true,
        bordertop=true,
        borderbottom=true,
        labeltoplift=0.4em,
        labeltop=false,
        labelbottom=true,
        labelleft=true,
        labelright=false,
        % printarea=a1-h5,
        %markfields=\fields,
        %markareas=\areas,
        setpieces={Ke7, Qh4},
        addpieces={kg2, ph2}
    ]
\end{nscenter}
\vspace{4pt}

\task Могут ли белые выиграть? Ход белых.
\begin{nscenter}
    \chessboard[
        showmover=true,
        borderleft=true,
        bordertop=true,
        borderbottom=true,
        labeltoplift=0.4em,
        labeltop=false,
        labelbottom=true,
        labelleft=true,
        labelright=false,
        % printarea=a1-h5,
        %markfields=\fields,
        %markareas=\areas,
        setpieces={Ke7, Qh4},
        addpieces={kg2, pf2}
    ]
\end{nscenter}
\vspace{4pt}


\task Могут ли черные выиграть? Ход черных.
\begin{nscenter}
    \chessboard[
        showmover=true,
        mover=b,
        borderleft=true,
        bordertop=true,
        borderbottom=true,
        labeltoplift=0.4em,
        labeltop=false,
        labelbottom=true,
        labelleft=true,
        labelright=false,
        % printarea=a1-h5,
        %markfields=\fields,
        %markareas=\areas,
        setpieces={Ke3, Bc3},
        addpieces={ke5, rd4, rd8}
    ]
\end{nscenter}
\vspace{4pt}

\task Могут ли белые выиграть, если их ход? А если ход черных?
\begin{nscenter}
    \chessboard[
        showmover=false,
        % mover=b,
        borderleft=true,
        bordertop=true,
        borderbottom=true,
        labeltoplift=0.4em,
        labeltop=false,
        labelbottom=true,
        labelleft=true,
        labelright=false,
        % printarea=a1-h5,
        %markfields=\fields,
        %markareas=\areas,
        setpieces={Kf3, Qe1},
        addpieces={kh2, rg2}
    ]
\end{nscenter}
\vspace{4pt}

\task Могут ли белые выиграть, если их ход? А если ход черных?
\begin{nscenter}
    \chessboard[
        showmover=false,
        % mover=b,
        borderleft=true,
        bordertop=true,
        borderbottom=true,
        labeltoplift=0.4em,
        labeltop=false,
        labelbottom=true,
        labelleft=true,
        labelright=false,
        % printarea=a1-h5,
        %markfields=\fields,
        %markareas=\areas,
        setpieces={Kh1, Qe6},
        addpieces={kf8, rg7}
    ]
\end{nscenter}
\vspace{4pt}

\task Могут ли белые выиграть, если их ход? А если ход черных?
\begin{nscenter}
    \chessboard[
        showmover=false,
        % mover=b,
        borderleft=true,
        bordertop=true,
        borderbottom=true,
        labeltoplift=0.4em,
        labeltop=false,
        labelbottom=true,
        labelleft=true,
        labelright=false,
        % printarea=a1-h5,
        %markfields=\fields,
        %markareas=\areas,
        setpieces={Ke2, Pf7},
        addpieces={kh4, rg1}
    ]
\end{nscenter}
\vspace{4pt}


\task Могут ли белые выиграть?
\begin{nscenter}
    \chessboard[
        showmover=true,
        % mover=b,
        borderleft=true,
        bordertop=true,
        borderbottom=true,
        labeltoplift=0.4em,
        labeltop=false,
        labelbottom=true,
        labelleft=true,
        labelright=false,
        % printarea=a1-h5,
        %markfields=\fields,
        %markareas=\areas,
        setpieces={Kh5, Re6, Pf5, Pg6},
        addpieces={kf1, rf8, bf6}
    ]
\end{nscenter}
\vspace{4pt}

\task Найдите мат в 3 хода за белых.
\begin{nscenter}
    \chessboard[
        showmover=true,
        % mover=b,
        borderleft=true,
        bordertop=true,
        borderbottom=true,
        labeltoplift=0.4em,
        labeltop=false,
        labelbottom=true,
        labelleft=true,
        labelright=false,
        % printarea=a1-h5,
        %markfields=\fields,
        %markareas=\areas,
        setpieces={Kg1, Qh6, Rf1, Rf3, Pe4, Pd4, Pg2},
        addpieces={kg8, qe8, rd8, rh8, bg4, pd6, pg6}
    ]
\end{nscenter}
\vspace{4pt}

\task Найдите выигрыш за белых.
\begin{nscenter}
    \chessboard[
        showmover=true,
        % mover=b,
        borderleft=true,
        bordertop=true,
        borderbottom=true,
        labeltoplift=0.4em,
        labeltop=false,
        labelbottom=true,
        labelleft=true,
        labelright=false,
        % printarea=a1-h5,
        %markfields=\fields,
        %markareas=\areas,
        setpieces={Kh7, Bh4, Pg6},
        addpieces={ke4, bf8}
    ]
\end{nscenter}
\vspace{4pt}

% \setchessboard{
%     inverse=false,
%     normalboard,
%     color=magenta,
%     clearboard,
%     showmover=true}


\newpage

\task Какое минимальное количество ходов необходимо белым, чтобы поставить мат? Ход белых.

\begin{nscenter}
    \chessboard[
        showmover=true,
        setpieces = {Kg1, Ba1, Nf6, Ph6},
        addpieces={kh8, bb7, pg3},
        pgfstyle= {[base,at={\pgfpoint{0pt}{-0.4ex}}]text},
        text= \fontsize{1.7ex}{1.2ex}\bfseries \sffamily\currentwq,
        markfields={},
        markareas={}
    ]
\end{nscenter}

\taskk Подумайте, могут ли черные из предыдущей задачи каким-то образом отсрочить мат?

\taskkk Если справились с предыдущей задачей, подумайте еще вот над чем:
могут ли белые из предыдущей задачи, несмотря на оттягивание мата со стороны черных,
все же немного его ускорить?

\taskkk Ситуация кажется для белых безысходной:
у черных преимущество в ладью и пешки вот-вот пройдут в ферзи.
Могут ли белые как-то спасти свое положение? Ход белых.

\begin{nscenter}
    \chessboard[
        showmover=true,
        setpieces = {Kd6, Qb6, Bc7, Pb4, Pc5},
        addpieces={ka8, qb1, rc8, bb5, pa6, pc6, pd3, pe2},
        pgfstyle= {[base,at={\pgfpoint{0pt}{-0.4ex}}]text},
        text= \fontsize{1.7ex}{1.2ex}\bfseries \sffamily\currentwq,
        markfields={},
        markareas={}
    ]
\end{nscenter}
\vspace{4pt}

\newpage

\taskkk Найдите мат в два хода за белых.

\begin{nscenter}
    \chessboard[
        showmover=true,
        setpieces = {Kg7, Qf7, Rb5, Re1},
        addpieces={ka8, rf5, bg6, nd6, pb7, pd5, pf4, pf6},
        pgfstyle= {[base,at={\pgfpoint{0pt}{-0.4ex}}]text},
        text= \fontsize{1.7ex}{1.2ex}\bfseries \sffamily\currentwq,
        markfields={},
        markareas={}
    ]
\end{nscenter}
\vspace{4pt}



\end{document}