\documentclass{article}
\usepackage[utf8]{inputenc}
\usepackage[russian]{babel}
\usepackage{geometry}
 \geometry{
 a4paper,
 total={170mm,257mm},
 left=20mm,
 top=20mm,
 }
\usepackage{caption}

\usepackage{skak}
\usepackage{xskak}
\usepackage{chessboard}

\captionsetup[figure]{labelfont={small,it},textfont={small,it}}

\addto\captionsenglish{%
    \renewcommand{\figurename}{}}

\setlength{\abovecaptionskip}{-10pt plus 3pt minus 2pt}

\newenvironment{nscenter}
 {\parskip=0pt\par\nopagebreak\centering}
 {\par\noindent\ignorespacesafterend}

\newcounter{taskno}
\stepcounter{taskno}
\newcounter{subtaskno}[taskno]

\newcommand{\task}[0]{\par\noindent\textbf{Задача\,\arabic{taskno}.} \stepcounter{taskno} \stepcounter{subtaskno}}
\newcommand{\taskk}[0]{\par\noindent\textbf{Задача\,\arabic{taskno}*.} \stepcounter{taskno} \stepcounter{subtaskno}}

\begin{document}

\task Найдите мат в два хода за белых.
\begin{nscenter}
    \chessboard[
        showmover=true,
        mover=w,
        setpieces = {Kg1, Qd5, Ra1, Rf1, Bb3, Be3, Ng5, Pa2, Pb2, Pc3, Pf2, Pg2, Ph2},
        addpieces={kh8, qd8, ra8, rf8, be7, bf5, nc6, ne4, pa6, pb5, pc7, pg7, ph7},
        pgfstyle= {[base,at={\pgfpoint{0pt}{-0.4ex}}]text},
        text= \fontsize{1.7ex}{1.2ex}\bfseries \sffamily\currentwq,
        markfields={},
        markareas={}
    ]
\end{nscenter}
\vspace{4pt}

\task Найдите выигрыш за черных.
\begin{nscenter}
    \begin{figure}[h]
        \centering
        \caption*{из партии Флор — Толлуш\\Таллин, 1945 г.}
        \chessboard[
            showmover=true,
            mover=b,
            setpieces = {Kg2, Rc1, Re4, Pb3, Pf2, Pf3, Ph2},
            addpieces={kf8, rd6, bg7, pa2, pc3, pf7, pg6, ph7},
            pgfstyle= {[base,at={\pgfpoint{0pt}{-0.4ex}}]text},
            text= \fontsize{1.7ex}{1.2ex}\bfseries \sffamily\currentwq,
            markfields={},
            markareas={}
        ]
    \end{figure}
\end{nscenter}
\vspace{4pt}

\task Найдите ничью за белых.
\begin{nscenter}
    \chessboard[
        showmover=true,
        mover=w,
        setpieces = {Kd2, Bd1, Bd8, Pa2, Pb2, Pc3, Pd4, Pe5, Pf4, Pg4, Ph3},
        addpieces={kb5, ra6, rb7, ba5, pa3, pb4, pc5, pd6, pe7, pf6, pg5, ph4},
        pgfstyle= {[base,at={\pgfpoint{0pt}{-0.4ex}}]text},
        text= \fontsize{1.7ex}{1.2ex}\bfseries \sffamily\currentwq,
        markfields={},
        markareas={}
    ]
\end{nscenter}
\vspace{4pt}


\task Старинная задача. Найдите мат в три хода за белых.
\begin{nscenter}
    \chessboard[
        showmover=true,
        mover=w,
        setpieces = {Kc1, Nb4, Nd4, Pe2},
        addpieces = {ka1, bb1, bh6, pc4, pc3, pc2, pe3},
        pgfstyle= {[base,at={\pgfpoint{0pt}{-0.4ex}}]text},
        text= \fontsize{1.7ex}{1.2ex}\bfseries \sffamily\currentwq,
        markfields={},
        markareas={}
    ]
\end{nscenter}
\vspace{4pt}

\task Найдите выигрыш за белых.
\begin{nscenter}
    \begin{figure}[h]
        \centering
        \caption*{Керес — Файн\\Матч СССР — США, 1946 г.}
        \chessboard[
            showmover=true,
            mover=w,
            setpieces = {Kg1, Qh3, Rf1, Rd1, Bf4, Ng5, Pd5, Pf2, Pg2, Ph2},
            addpieces={kg8, qb5, rc8, rf8, bg7, nf6, pc5, pf7, pg6, ph7},
            pgfstyle= {[base,at={\pgfpoint{0pt}{-0.4ex}}]text},
            text= \fontsize{1.7ex}{1.2ex}\bfseries \sffamily\currentwq,
            markfields={},
            markareas={}
        ]
    \end{figure}
\end{nscenter}

\task Найдите выигрыш за черных.
\begin{nscenter}
    \begin{figure}[h]
        \centering
        \caption*{Левенфиш — Чеховер\\Москва, 1935 г.}
        \chessboard[
            showmover=true,
            mover=b,
            setpieces = {Kh2, Qd4, Rg1, Bb2, Pa2, Pb4, Pc4, Pd6, Pf2, Ph4},
            addpieces={kg8, qh6, rf7, ng7, pa7, pb6, pd7, pe6, ph7},
            pgfstyle= {[base,at={\pgfpoint{0pt}{-0.4ex}}]text},
            text= \fontsize{1.7ex}{1.2ex}\bfseries \sffamily\currentwq,
            markfields={},
            markareas={}
        ]
    \end{figure}
\end{nscenter}

\taskk Попробуйте оценить позицию и найти за черных выигрышную стратегию.
\begin{nscenter}
    \chessboard[
        showmover=true,
        mover=b,
        setpieces = {Kb3, Be3, Pd4, Pf4, Pg5},
        addpieces={kb5, nc6, pd5, pf5, pg6},
        pgfstyle= {[base,at={\pgfpoint{0pt}{-0.4ex}}]text},
        text= \fontsize{1.7ex}{1.2ex}\bfseries \sffamily\currentwq,
        markfields={},
        markareas={}
    ]
\end{nscenter}
\vspace{4pt}

\taskk Данная позиция немного отличается от позиции из предыдущей задачи.
Есть ли здесь у черных стратегия для победы?
\begin{nscenter}
    \chessboard[
        showmover=true,
        mover=b,
        setpieces = {Kc3, Bf3, Pd4, Pf4, Pg5},
        addpieces={kb5, ne7, pd5, pf5, pg6},
        pgfstyle= {[base,at={\pgfpoint{0pt}{-0.4ex}}]text},
        text= \fontsize{1.7ex}{1.2ex}\bfseries \sffamily\currentwq,
        markfields={},
        markareas={}
    ]
\end{nscenter}
\vspace{4pt}

\taskk Попробуйте оценить позицию и подумать над выигрышной стратегией для черных.
\begin{nscenter}
    \chessboard[
        showmover=true,
        mover=b,
        setpieces = {Ke2, Nd5, Pa4},
        addpieces={ka5, bg1, pa6, pe4},
        pgfstyle= {[base,at={\pgfpoint{0pt}{-0.4ex}}]text},
        text= \fontsize{1.7ex}{1.2ex}\bfseries \sffamily\currentwq,
        markfields={},
        markareas={}
    ]
\end{nscenter}
\vspace{4pt}

\end{document}